\chapter*[Introdução]{Introdução}
\addcontentsline{toc}{chapter}{Introdução}

O Brasil é composto atualmente por 26 Estados (Unidades Federativas), 01 Distrito Federal e 5.565 Municípios.
As principais entidades da administração pública são o Governo Federal, Governos Estaduais, Governos Municipais, Sociedades de Economia Mista, Fundações, Autarquias, e Empresas Estatais~\cite{thesing2012municipios}.

A licitação pública é um processo administrativo por meio do qual a administração pública seleciona a proposta mais lucrativa, menos onerosa e com melhor qualidade possível, para a contratação de obras, serviços, compras de produtos, locações ou alienações~\cite{l2300}.

Todas as entidades federativas, União, Estados, Municípios e Distrito Federal, são obrigados a licitar. 
Porém cada um possui sua própria competência para legislar, decorrente de sua autonomia política e administrativa~\cite{l8666}.

As leis que regem as licitações são as Leis nº 8.666 de 1993 e nº 10.520 de 2002. 
Essas duas leis permitem que os governos façam seus regulamentos próprios, isso para facilitar e adequar as regras gerais às particularidades de cada administração pública~\cite{123}.
Entretanto, qualquer Lei estadual, Decreto ou Regulamento está subordinado a Lei nº 8.666 de 1993 e a Lei nº 10.520 de 2002~\cite{123}. 

Um dos objetivos da Lei nº 8.666 de 1993 é garantir a observância do princípio constitucional da isonomia, a seleção da proposta mais vantajosa para a administração e a promoção do desenvolvimento nacional sustentável~\cite{l8666}.

Ela deve estar em estrita conformidade com os princípios básicos da legalidade, da impessoalidade, da moralidade, da igualdade, da publicidade, da probidade administrativa, da vinculação ao instrumento convocatório, do julgamento objetivo e dos que lhes são correlatos~\cite{l8666}.

As modalidades inicialmente da Lei nº 8.666 de 1993 são cinco: licitação de pregão presencial, carta convite, tomada de preço, concorrência pública e concurso. 
A licitação por pregão eletrônico foi criada anos depois através da Lei nº 10.520 em 2002~\cite{l10529}.

A licitação se inicia por uma necessidade da administração pública. 
A primeira fase é interna e é dedicada ao planejamento. 
A fase externa inicia-se com a publicação da licitação e termina com o objetivo principal, o contrato. 
Na fase do Contrato cabe à contratada executar e à administração fiscalizar a execução~\cite{camarao2013fase}.

Há muitos detalhes em cada modalidade licitatória descritos em legislação complementar como os Decretos nº 3.555 de 2000 e nº 5.450 de 2005, e Decreto-Lei nº 2.300 de 1986~\cite{l3555},~\cite{l5450}.

O IFG é uma rede pública de educação superior, básica e profissional, pluricurricular e \textit{multicampi}, especializada em ofertar educação profissional e tecnológica nas diferentes categorias de ensino, com suporte na união de conhecimentos técnicos e tecnológicos~\cite{apresentacao}.

Sendo uma rede pública, o IFG possui áreas onde ficam localizados esses Institutos Federais.
Esses locais são chamados de câmpus.

Durante a fase interna, aquela dedicada ao planejamento, quando um dos câmpus inicia um processo de licitação, é preciso fazer uma consulta aos demais \textit{campi}~\cite{carvalho2013manual}. 

O objetivo desta consulta é determinar a demanda de cada câmpus para cada item licitado. 
Problemas de comunicação podem comprometer esse processo e muitos \textit{campi} podem ficar fora da licitação~\cite{fernandes2005politicas}.

Ainda durante a fase interna, existe o período de solicitação de documentação para a consolidação de dados, que é pedido pela procuradoria federal~\cite{de2007curso}. 

Esse processo leva um tempo médio de um ano e meio, de acordo com setor de compras do IFG câmpus Formosa.
A comunicação entre os \textit{campi} em um processo de licitação é uma demanda real e urgente para garantir o Princípio Constitucional da Eficiência~\cite{l1988}.

\subsection*{Problema}
Ausência de um meio eficiente de comunicação inter-\textit{campi} para alcançar o Princípio Constitucional da Eficiência na fase interna de processos licitatórios no IFG.

%O IFG foi usado como estudo de caso para identificar o problema que não há um sistema integrado entre \textit{campi} com vistas a melhorar a comunicação em processos licitatórios.

\subsection*{Objetivo}

Construir um sistema de comunicação inter-\textit{campi} para melhorar a fase interna de processos licitatórios no IFG.

\subsubsection*{Objetivos Específicos}
\begin{itemize}
    \item Criar um sistema, através do qual as partes interessadas poderão estabelecer rotinas legais e bem definidas abrangendo o máximo de demandas de compras em um único processo licitatório.
    \item Apoiar as tomadas de decisão, a inteligência organizacional e a segurança da informação com vistas à redução de erros, custos e desperdícios, e ao aumento da precisão e produtividade.
\end{itemize}

O Capítulo~\ref{Capitulo1} aborda historicamente leis e decretos brasileiros que definem o processo licitatório.
O Capítulo~\ref{Capitulo2} aborda a fase interna da licitação e descreve o problema existente.
O Capítulo~\ref{Capitulo3} identifica e define as ferramentas e tecnologias usadas e as razões pelas quais foram escolhidas.
O Capítulo~\ref{Capitulo4} descreve a aplicação das metodologias, ferramentas e tecnologias de desenvolvimento de software usadas para a criação da aplicação.
O Capítulo~\ref{Capitulo5} mostra quais foram os resultados obtidos, enquanto o Capítulo~\ref{Capitulo6} apresenta as conclusões.

