\part{Processo Licitatório}

\chapter[Processo Licitatório no Brasil]{Processo Licitatório no Brasil}\label{Capitulo1}

Este capítulo apresenta uma visão geral sobre o processo licitatório no Brasil abrangendo seu histórico desde 1862 até o processo contemporâneo previsto na legislação em vigor.

\section{Histórico}

Licitação é um sistema de compras públicas que tem o objetivo de preservar a transparência e integridade das tais.
Esse modelo existia desde a Roma Antiga de VIII a.C. e foi aplicado para a comercialização de espólio das guerras~\cite{prestes2004guia}. 

Nesta caso as mercadorias ficavam a mostra em meios a lanças fincadas no chão, e os interessados faziam a oferta de compras para os interessados. 
Gradualmente, esse processo foi ampliado aos contratos públicos para a compra de produtos pelo governo~\cite{prestes2004guia}.

\subsection{No Brasil}

No Brasil império iniciou-se o processo licitatório com o Decreto nº 2.926 de 1862, que regulamentava as compras dos serviços do então Ministério da Agricultura, Comércio e Obras Públicas~\cite{de8processo}. 

Durante o Brasil República, onde a sociedade mostrava preocupação com os gastos públicos, foi se criado diversas leis tratavam sobre o assunto, houve então um esforço de estabelecer normas relativas ao procedimento licitatório em âmbito federal, que foi feito através do Código de Contabilidade da União Decreto nº 4.536 de 1922~\cite{oliveira2013}.

Contudo houve ainda várias mudanças e melhorias no método licitatório tendo o objeto de dar maior eficiência às contratações públicas.
Houve uma sistematização do tema com o Decreto-Lei nº 200 de 1962, que determinava uma reforma administrativa no âmbito federal, que foi em seguida estendida para às Administrações dos Estados e Municípios pela Lei nº 5.45 de 1968~\cite{ribeiro2007evoluccao}.

Mesmo com todo este processo de organização da Licitação na legislação, ainda havia muitas lacunas que oferecia interpretações diferentes, que era vindo de uma administração centralizadora e burocrática~\cite{filho2013}. 
%E por meio do processo de redemocratização, houve uma tentativa de governar de uma nova forma~\cite{filho2013}.

Nesse intervalo, o processo licitatório conseguiu popularidade por meio do Decreto-Lei nº 2.300 de 1986, sendo atualizada em 1987, com os Decretos–lei nº 2.348 e 2.360, que pela primeira vez criavam um Estatuto Jurídico das Licitações e Contratos Administrativos, junto a normas gerais e especiais associados aos conteúdos~\cite{oliveira2013}.

O Estatuto Jurídico das Licitações e Contratos Administrativos, ao lado da Constituição de Federal de 1988, configurou um importante desenvolvimento na institucionalização e democratização da Administração Pública.

Com a Constituição de Federal de 1988, a licitação ganhou um \textit{status} de princípio constitucional, de observância obrigatória pela Administração Pública direta e indireta de todos os entes da Federação: União, Estados, Distrito Federal e Municípios~\cite{de2011pregao}.

Por tanto, mesmo com a troca de governos o que influenciava a mudança das leis era a sociedade, devido aos seus anseios de controle de gastos e pelo próprio processo evolutiva da mesma, portanto o processo licitatório se tornava cada vez mais relevante.~\cite{brazao2013}.

\section{Definição}

A palavra licitação vem do latim \textit{licitatione}, que significa vendo por lances. 
O ``Ato ou efeito de licitar é uma disputa entre a importância da licitação para a administração pública os proponentes, em leilão, hasta pública, partilha judicial, ou concorrência administrativa, para alienação ou adjudicação ao autor da melhor proposta''~\cite{dicionario}.

Segundo~\cite[p. 495]{justen2010curso}, a licitação é um procedimento administrativo disciplinado por lei e por um ato administrativo prévio, que determina critérios objetivos visando a seleção da proposta de contratação mais vantajosa e a promoção do desenvolvimento nacional sustentável, com observância do princípio da isonomia, conduzido por um órgão dotado de competência especifica.

De acordo com~\cite[p.519]{mello2009celso}, licitação é um procedimento administrativo pelo qual uma pessoa governamental, pretendendo alienar, adquirir ou locar bens, realizar obras ou serviços, outorgar concessões, permissões de obra, serviço ou de uso exclusivo de bem público, segundo condições por ela estipuladas previamente, convoca interessados na apresentação de propostas, a fim de selecionar a que se revele mais conveniente em função de parâmetros antecipadamente estabelecidos e divulgados.

Para~\cite[p.576]{alexandrino2008direito}, licitação é um procedimento administrativo que deve ser obrigatoriamente observado pelas entidades governamentais, onde, garantida a igualdade entre os participantes, deve ser selecionada a melhor proposta dentre as oferecidas pelos interessados em travar determinadas relações de conteúdo patrimonial com o Poder Público, uma vez preenchidos os requisitos mínimos necessários para o bom cumprimento das obrigações que eles se propõem a executar.

\cite[p.356]{di2006500}, por sua vez, define que a licitação pode ser definida como um procedimento administrativo através do qual um ente público, fazendo-se valer do seu exercício da função administrativa, abre a todos os interessados, que se enquadrem nas condições fixadas no instrumento convocatório, a possibilidade de oferecerem propostas dentre as quais será selecionada e aceita a mais conveniente para a celebração do contrato.

Portanto, nota-se que licitação não é apenas um procedimento da Administração Pública, mas deve ser tratado também como um procedimento administrativo essencial.

Logo o conceito dado pelos autores fica da seguinte forma a, licitação sendo um procedimento administrativo da Administração Pública, que visa adquirir bens ou serviços, através da seleção da proposta mais benéfica à Administração Pública, proporcionando igualdade nas condições dos que participem, buscando através do Contrato Administrativo impulsionar os ganhos da coletividade.

\section{Processo Licitatório Contemporâneo}

A Constituição Federal de 1988 é de suma importância para o processo licitatório contemporâneo, porque com o Art. 22 inciso XXVII, foi estabelecido que compete privativamente à União Legislar enquanto o inciso XVII estabelece normais gerais de licitação e contratação em todas as modalidades para a Administração Pública.

A Lei nº 8.666 de 1993 foi um caso de atualização para melhorar e efetivar normas gerais para licitações e contratos da Administração Pública, no âmbito federal tornando-se um marco na aplicação do processo licitatório no país.
Em seus Artigos 22 e 23, da Lei nº 8666 de 1993 prevê inicialmente cinco modalidades de licitação, sendo: a concorrência, tomada de preços, convite, leilão e concurso.

\begin{itemize}
    \item \textbf{concorrência} está prevista no inciso I do Art. 22 da Lei 8.666 de 1993, como sendo, licitação entre quaisquer interessados que, na fase inicial de habilitação preliminar, comprovem possuir os requisito mínimos de qualificação exigidos no edital para execução de seu objeto.

    \item \textbf{tomada de preço} está prevista no inciso II do Art. 22 da lei 8.666 de 1993, como sendo, licitação entre interessados devidamente cadastrados ou que atenderem a todas as condições exigidas para cadastramento até o terceiro dia anterior à data do recebimento das propostas, observada a necessária qualificação.

    \item \textbf{convite} está previsto no inciso III do Art. 22 da lei 8.666 de 1993, como sendo, licitação entre interessados do ramo pertinente ao seu objeto, cadastrados ou não, escolhidos e convidados em número mínimo de três pela unidade administrativa, a qual afixará, em local apropriado, cópia do instrumento convocatório e o estenderá aos demais cadastrados na correspondente especialidade que manifestarem seu interesse com antecedência de até vinte e quatro horas da apresentação das propostas. 

    \item \textbf{concurso} está previsto no inciso IV do Art. 22 da lei 8.666 de 1993, como sendo, licitação entre quaisquer interessados para escolha de trabalho técnico, científico e artístico, mediante a instituição de prêmios ou remuneração aos vencedores, conforme critérios constantes de edital publicado na imprensa oficial com antecedência mínima de quarenta e cinco dias. 

    \item \textbf{leilão} está previsto no inciso V do Art. 22 da lei 8.666 de 1993, como sendo, licitação entre quaisquer interessados para a venda de bens móveis inservíveis para a Administração ou de produtos legalmente apreendidos ou penhorados, ou para a alienação de bens imóveis prevista no Art. 19, a quem oferecer o maior lance, igual ou superior ao valor da avaliação. 
\end{itemize}

Com intuito de tornar o processo licitatório menos burocrático para alguns casos, uma solução adotada foi a Medida Provisória nº 2.026 de 2000, em que fora instituída a nova modalidade licitatória de Pregão. 
Essa medida provisória, foi transformada na Medida Provisória nº. 2.182 de 2001, que por sua vez foi reeditada sucessivamente por 18 vezes~\cite{ribeiro2007evoluccao}. 

O Pregão no começo funcionava apenas para a União mas através da promulgação da Lei Federal nº 10.520 de 2002, estendeu a aplicação do Pregão modalidade também aos Estados e Municípios~\cite{ribeiro2007evoluccao}.

\begin{itemize}
    \item \textbf{pregão} é a modalidade de licitação para aquisição de bens e serviços comuns, onde a disputa pelo fornecimento se dá através de sessão pública, presencial ou eletrônica (via internet), por meio de propostas e lances, para a classificação e habilitação do licitante que ofertou o menor preço~\cite{barreto2008licitaccoes}.
\end{itemize}

Diferentemente das outras modalidades de licitação, no pregão, a escolha da proposta é anterior à análise da documentação, razão pela qual o processo é mais célere. 

Ao invés de se analisar toda a documentação de todos os licitantes, analisa-se somente a documentação de quem apresentou a melhor proposta, estando em conformidade com o que fora solicitado no edital, o processo estará encerrado, caso a documentação não esteja em conformidade com o edital, analisar-se-á a documentação de quem apresentou a segunda melhor proposta, e assim sucessivamente~\cite{gasparini2017direito}.

Com isso viabilizou-se uma diminuição no tempo médio das licitações (de 50\%, reduzindo de 35 dias, em média, para 17)
e nos preços praticados pelos licitantes (de 25\%, em média)~\cite{mello2016licitaccao}. 

Mas logo em seguida foi-se criado uma melhoria do pregão que é o pregão eletrônico que é uma melhoria do pregão tradicional, esse surgimento se dá por causa da melhoria da informatização, mas ele não possui uma regulamentação na Lei nº 10.520 de 2002, ele é apenas regulamento pelo Decreto nº 5.450 de 2005.~\cite{dapregao}.

Os avanços tecnológicos no campo das comunicações criaram uma ampliação da possibilidade de acesso aos leilões que podem serem realizados pela internet, permitindo a empresas de todo o país participem com custos reduzidos.
Além disso, os leilões de lances regressivos possibilitam que os licitantes reduzam ao mínimo seus preços a fim de ganharem os contratos~\cite{mello2016licitaccao}.

O pregão eletrônico também diminuiu o prazo médio das licitações por sua sistemática de inversão de fases, deixando para verificar a documentação de habilitação ao final, uma vez conhecido o ganhador da proposta de preço.
Isso evita recursos que geram burocracia. 
O próprio funcionamento eletrônico também ajuda a ganhar tempo, uma vez que as atas e diversos outros documentos são gerados automaticamente pelo software utilizado~\cite{mello2016licitaccao}. 


\chapter[Fase Interna do Processo Licitatório]{Fase Interna do Processo Licitatório}\label{Capitulo2}

A fase interna advem da necessidade de contratação ou aquisição de serviço ou de obra.
Ela é constituída por um conjunto de atos e atividades preparatórias, as quais ficam a cargo do órgão administrativo.
Todas essas providências administrativas antecedem a realização das atividades que contam com a participação de terceiros~\cite{meirelles2011direito}.

A primeira atividade a ser feita é justificar a necessidade da contratação e definir o objetivo da competição.
São exigidos vários atos que são tomados como critérios para a aceitação de eventuais propostas, tais como a habilitação das potenciais empresas fornecedoras, a antecipação das cláusulas contratuais com a necessária fixação do prazo de fornecimento, e a avaliação prévia do bem ou serviços a serem contratados~\cite{carvalho2013manual}.

Esses atos e atividades são vistos de acordo com a autoridade competente do órgão da Administração Pública, a quem cabe a tarefa de gerenciar todos os trabalhos que estão em disputa e receber as propostas, lances e analisar a sua aceitabilidade de acordo com o Art. 3º da Lei Nº 10.520 de 2002~\cite{fernandes2005politicas}.

Na fase interna, no período da definição do objeto, o qual servirá como a base para o edital da licitação, frequentemente ocorrem equívocos que não podem ser solucionados posteriormente e que acabam gerando um transtorno para o procedimento a seguir~\cite{camarao2017fase},~\cite{de2007curso}. 
A fase interna da licitação é, portanto, fundamental e merece importância.

O sucesso na fase interna está ligado a princípios de eficiência e da economicidade.
Os atuantes nesse processo, devem ser capacitados para melhor especificar o objeto da contratação gerando maior possibilidade de garantir as metas da administração~\cite{camarao2017fase}.