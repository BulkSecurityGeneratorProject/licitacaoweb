\begin{resumo}
 A licitação pública é um processo administrativo por meio do qual a administração pública realiza a contratação de obras, serviços, compras de produtos, locações ou alienações.
 As modalidades de licitação são seis, licitação de pregão presencial, carta convite, tomada de preço, concorrência pública, concurso previstas na Lei 8.666 de 1993 e o pregão eletrônico criado anos depois através da Lei 10.520 em 2002.
 A primeira fase da licitação é interna e é dedicada ao planejamento.
 Em órgãos públicos multi-\textit{campi}, o processo de comunicação na fase interna pode apresentar dificuldades.
 Neste trabalho identificamos a ausência de um meio eficiente de comunicação inter-\textit{campi} para alcançar o Princípio Constitucional da Eficiência na fase interna de processos licitatórios no IFG.
 Construímos então, um sistema de comunicação inter-\textit{campi} para melhorar a fase interna de processos licitatórios no IFG.
 O resultado é uma aplicação web 2.0 que ajuda na fase interna de licitação através de uma melhoria de comunicação.
 A aplicação mantém uma base de dados sempre atualizada através da API de compras governamentais.
 Através dela, o usuário pode buscar item de interesse pelo seu código ou descrição, ou ainda aderir a compras de outros \textit{campi}.
 Esta solução agiliza um processo burocrático, dando celeridade na troca de informações sobre compras inter-\textit{campi}.
 \vspace{\onelineskip}
    
 \noindent
 \textbf{Palavras-chaves}: licitação, fase interna, comunicação, sistema.
\end{resumo}
