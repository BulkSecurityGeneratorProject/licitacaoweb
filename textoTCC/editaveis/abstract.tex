\begin{resumo}[Abstract]
 \begin{otherlanguage*}{english}
Public bidding is an administrative process through which the public administration performs the contracting of works, services, purchases of products, leases or disposals.
The bidding modalities are six, in loco public bidding, invitation letter, best price, public competition and presential, provided for in Law 8,666 of 1993 and the electronic trading session created years later through Law 10,520 in 2002.
The first phase of the bidding is internal and is dedicated to planning.
In public multi-campuses, the internal phase communication process may present problems.
In this work, we have identified the absence of an efficient means of inter-textual communication to reach the Constitutional Principle of Efficiency in the internal phase of bidding processes in the IFG.
We then constructed an inter-campus communication system to improve the internal phase of bidding processes in the IFG.
The result is a modern web application that helps in the internal bidding phase through a communication improvement.
The application maintains a database always updated through the API of governmental bidding.
Through it, the user can search for an item of interest by its code or description, or join the purchases of other campuses.
This solution streamlines a bureaucratic process, speeding up the exchange of information on inter-campus bidding.

   \vspace{\onelineskip}
 
   \noindent 
   \textbf{Key-words}: bidding, internal phase, communication, system.
 \end{otherlanguage*}
\end{resumo}
