\chapter[Conclusão]{Conclusão}\label{Capitulo6}

\section{Conclusão}

Foi desenvolvida uma aplicação web que tem o objetivo de melhorar a comunicação na fase interna de um processo de licitação.
O sistema foi desenvolvido a partir do levantamento de dados do IFG, entretanto é facilmente adaptado para qualquer instituição que opte por usar essa ferramenta. 

Utilizamos modernas técnicas e metodologias, as quais também foram adequadas para o cenário encontrado, que demandou comprometimento com uma solução viável em curto prazo.
O sistema foi desenvolvido para web para maximizar sua usabilidade independente do sistema operacional das instituições interessadas.
A solução é construída sobre dados de licitações, itens, materiais e serviços obtidos do Governo Federal, mantendo-se atualizada por meio da API de compras governamentais.

A construção dessa aplicação oportunizou-nos aplicar conceitos, técnicas e métodos relacionados à análise e desenvolvimento de sistemas, aprendidos ao longo do curso.
A implantação do sistema deve ocorrer tão logo o setor de T.I. do câmpus IFG-Formosa seja demandado pelos gestores.

Para trabalho futuro, pretendemos implementar uma consulta aberta às licitações.
Tal consulta permitirá à comunidade acesso às licitações com os itens e valores, filtrando por órgão público, cidade ou estado. 
Dessa forma o sistema contribuirá também para o controle social dos atos públicos.